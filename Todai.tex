\documentclass[a4paper,fleqn,dvipdfmx]{jsarticle}
\usepackage[utf8]{inputenc}
\usepackage{bm}
\usepackage{tikz}
\usepackage{multicol}
\usepackage{otf}
\setlength{\mathindent}{2zw}
\usepackage{empheq}
\setlength{\textwidth}{\fullwidth}
\setlength{\textheight}{41\baselineskip}
\addtolength{\textheight}{\topskip}
\setlength{\voffset}{-0.5in}
\setlength{\headsep}{0.3in}
\setlength{\topmargin}{0.3in}
\setcounter{page}{1}
\usepackage{color}
\usepackage{ascmac}
\usepackage{amssymb, amsmath}
\usepackage{tcolorbox}
\usepackage{fancyhdr}
\pagestyle{fancy}
\lhead{数学100問} 
\chead{}
\rhead{}
\lfoot{}
\cfoot{-\thepage-}
\rfoot{}
\renewcommand{\headrulewidth}{0pt}
\begin{document}

\begin{flushleft}
【1】
\end{flushleft}
$1\leq m\leq n$となる正整数を考えたとき、$m$は次の値を割り切ることを示せ
$$n\sum_{k=0}^{m-1}(-1)^k\binom{n}{k}
$$
ただし
$$\binom{n}{k}=_n\mathrm{C}_k$$
\begin{flushleft}
【解答】
\end{flushleft}
\begin{eqnarray}
n\sum_{k=0}^{m-1}(-1)^k\binom{n}{k}&=&n\sum_{k=0}^{m-1}(-1)^k\left(\binom{n-1}{k}+\binom{n-1}{k-1}\right)\\
&=&n\sum_{k=0}^{m-1}(-1)^k\binom{n-1}{k}-n\sum_{k=0}^{m-2}(-1)^k\binom{n-1}{k-1}\\
&=&n(-1)^{m-1}\binom{n-1}{m-1}=m(-1)^{m-1}\binom{n}{m}
\end{eqnarray}

\begin{flushleft}
【2】
\end{flushleft}
パスカルの三角形の7番目の部分には、7、21、35という連続した二項係数があり、これが等差数列をなしている。3項の等差数列を含む行をすべて求めよ。
\begin{flushleft}
【解答】
\end{flushleft}
$$2\binom{n}{k}=\binom{n}{k-1}+\binom{n}{k+1}$$
$$4k^2-4nk+n^2-n-2=0$$
$$k=\frac{n\pm \sqrt{n+2}}{2}$$
$$n+2=m^2$$
$$k=\frac{(m+1)(m-2)}{2} or \frac{(m-1)(m=2)}{2}$$
7以上の平方数-2となる数となる

\begin{flushleft}
【3】
\end{flushleft}
等差数列$\{a_n\}$と$S_n=\sum a_n$に対して、次の式が成立することをしめせ

$$\sum_{k=0}^{n}\binom{n}{k}a_{k+1}=\frac{2^n}{n+1}S_{n+1}$$
\begin{flushleft}
【解答】
\end{flushleft}
$$(a_{k+1}+a_{n-k+1})(n+1)=2S_{n+1}$$
$$\sum_{k=0}^n\binom{n}{k}(a_{k+1}+a_{n-k+1}(n+1)=2S_{n+1}=\frac{2S_{n+1}}{n+1}\sum_{k=0}^{n}\binom{n}{k}=\frac{2^n}{n+1}S_{n+1}$$

\begin{flushleft}
【4】
\end{flushleft}
$p$を素数として、$n,k$の$p$進数表示の第$i$桁目を$a_i,b_i$とする。また、$i>r$で$a_i$は全て$0$となる。この時次のことを示せ。
$$\binom{n}{k}\equiv\prod_{i=0}^{r}\binom{a_i}{b_i}\ (\mathrm{mod}\ p)$$
次に、パスカルの三角形で、全ての数が奇数となる行の行数$m$の必要十分条件は$m=2^n-1$となる正整数$n$が存在することであることを証明せよ。
\begin{flushleft}
【解答】
\end{flushleft}
$$n=a_rp^r+...+a_1p+a_0$$
$$k=b_rp^r+...+b_1p+b_0$$
$$n_1=a_rp^{r-1}+...+a_1$$
$$k_1=b_rp^{r-1}+...+b_1$$
$\binom{n}{k}$の最初の$b_0$項を考え、また、分母分子ともに$p$の倍数ではないことを考えて、
$$(\frac{n_1p+a_0}{k_1p+b_0})(\frac{n_1p+a_0-1}{k_1p+b_0-1})...(\frac{n_1p+a_0-b_0+1}{k_1p+1})\equiv(\frac{a_0}{b_0})(\frac{a_0-1}{b_0-1})...(\frac{a_0-b_0+1}{1})=\binom{a_0}{b_0}\ (\mathrm{mod\ }p)$$
となることがわかる。以上から、
$$\binom{n}{k}\equiv\binom{a_0}{b_0}\binom{n_1}{k_1}\ (\mathrm{mod}\ p)$$
このような作業を$r$回続ければ証明は完了する。\\
次に、$p=2$の場合を考えれば、$\binom{n}{k}$全てが奇数となるのは、$a_i$が全て1の時のみなので、証明は完了する

\end{document}