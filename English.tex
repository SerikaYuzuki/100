\documentclass[b4paper,fleqn,dvipdfmx]{jsarticle}
\usepackage[utf8]{inputenc}
\usepackage{bm}
\usepackage{tikz}
\usepackage{multicol}
\usepackage{otf}
\setlength{\mathindent}{2zw}
\usepackage{empheq}
\setlength{\textwidth}{\fullwidth}
\setlength{\textheight}{41\baselineskip}
\addtolength{\textheight}{\topskip}
\setlength{\voffset}{-0.5in}
\setlength{\headsep}{0.3in}
\setlength{\topmargin}{0.3in}
\setcounter{page}{1}
\usepackage{color}
\usepackage{ascmac}
\usepackage{amssymb, amsmath}
\usepackage{tcolorbox}
\usepackage{fancyhdr}
\pagestyle{fancy}
\lhead{和訳100問} 
\chead{}
\rhead{}
\lfoot{}
\cfoot{-\thepage-}
\rfoot{}
\renewcommand{\headrulewidth}{0pt}
\begin{document}



\newpage

\begin{itembox}[l]{啓蒙}

    次の文章を和訳せよ。\\
    The aphorism, "As a man thinketh in his heart so is he," not only embraces the whole of a man’s being, but is so comprehensive as to reach out to every condition and circumstance of his life. A man is literally what he thinks, his character being the complete sum of all his thoughts.
    
\end{itembox}


\begin{flushleft}
【解答】
\end{flushleft}

    「人の心情こそ人格である」という金言は、個人の世界のみならず、周囲や環境にまで作用する。人生とはまさにどう思うであり、人格とは心の産物の蓄積なのだ。

\begin{flushleft}
【解説】
\end{flushleft}

諸手を挙げて添削などはできませんが、よかったら皆さんの和訳文が見てみたいです! そして、問題のないものを掲載させていただきたいです!



\newpage


\begin{itembox}[l]{啓蒙}

    次の文章を和訳せよ。\\
    Just as a gardener cultivates his plot, keeping it free from weeds, and growing the flowers and fruits which he requires, so may a man tend\footnote{take care of} the garden of his mind, weeding out all the wrong, useless, and impure thoughts, and cultivating toward perfection the flowers and fruits of right, useful, and pure thoughts.
    
\end{itembox}


\begin{flushleft}
【解答】
\end{flushleft}

庭師が花や実を育て上げるために、畑を耕して雑草を刈るように、人も精神の庭を耕し、誤謬や杞憂、雑念を刈り除いて初めて、正善で理性的かつ純粋な心情という花や実を育て上げるのだ。

\begin{flushleft}
【解説】
\end{flushleft}

Just as~ so...:~と同じように...\\
may S V(原型):SがVしますように(祈願文)\\
対比と比喩の多い文章です! 正確に文章を読み取りましょう! おそらくですが、文法事項にこだわった結果、文脈を読み違える事が起きているのかもしれません。\\
cultivating toward perfection \underline{the flowers and fruits} of right, useful, and pure thoughts.例えば下線部ですがこれは文章内には出ていない、文脈上の比喩で、「成果」に近い意味を持っています。



\newpage


\begin{itembox}[l]{自然科学}

    文章内の対比に言及しつつ、下線部の用語を説明せよ。\\
    What do we mean by “understanding” something? We can imagine that this complicated array of moving things\footnote{this mention to movements of atoms in nature} which constitutes “the world” is something like a great chess game being played by the gods, and we are observers of the game. We do not know what the rules of the game are; all we are allowed to do is to watch the playing. Of course, if we watch long enough, we may eventually catch on to a few of the rules. The rules of the game are what we mean by \underline{fundamental physics.}
    
\end{itembox}


\begin{flushleft}
【解答】
\end{flushleft}

自然界の法則への理解と、その法則の成立理由への理解は別物であり、自然界における観測者である人間は繰り返される事象の法則を見つけ出すことしかできないわけだが、その法則こそが物理学の基礎というものである。

\begin{flushleft}
【解説】
\end{flushleft}





\newpage


\begin{itembox}[l]{自然科学}

文章内の対比に言及しつつ、哲学者と物理学者の相対性理論の捉え方の違いを説明せよ。\\
Poincaré made the following statement of the principle of relativity: “According to the principle of relativity, the laws of physical phenomena must be the same for a fixed observer as for an observer who has a uniform motion of translation relative to him, so that we have not, nor can we possibly have, any means of discerning whether or not we are carried along in such a motion.”\\
Is it absolutely, definitely, philosophically\footnote{In this sentence, philosophers and physicists are written in contrast.} necessary that one should not be able to tell how fast he is moving without looking outside? One of the consequences of relativity was the development of a philosophy which said, “You can only define what you can measure! Since it is self-evident that one cannot measure a velocity without seeing what he is measuring it relative to, therefore it is clear that there is no meaning to absolute velocity. The physicists should have realized that they can talk only about what they can measure.” But that is the whole problem: whether or not one can define absolute velocity is the same as the problem of whether or not one can detect in an experiment, without looking outside, whether he is moving. In other words, whether or not a thing is measurable is not something to be decided a priori by thought alone, but something that can be decided only by experiment. Given the fact that the velocity of light is 186,000 mi/sec, one will find few philosophers who will calmly state that it is self-evident that if light goes 186,000 mi/sec inside a car, and the car is going 100,000 mi/sec, that the light also goes 186,000 mi/sec past an observer on the ground. That is a shocking fact to them; the very ones who claim it is obvious find, when you give them a specific fact, that it is not obvious.\\
There is even a philosophy which says that one cannot detect any motion except by looking outside. It is simply not true in physics. True, one cannot perceive a uniform motion in a straight line, but if the whole room were rotating we would certainly know it, for everybody would be thrown to the wall—there would be all kinds of “centrifugal” effects. That the earth is turning on its axis can be determined without looking at the stars, by means of the so-called Foucault pendulum, for example. Therefore it is not true that “all is relative”; it is only uniform velocity that cannot be detected without looking outside. Uniform rotation about a fixed axis can be. When this is told to a philosopher, he is very upset that he did not really understand it, because to him it seems impossible that one should be able to determine rotation about an axis without looking outside.

\end{itembox}


\begin{flushleft}
【解答】
\end{flushleft}

ポアンカレの解釈を曲解した結果、哲学者は相対性理論があらゆる事象が全て相対的であるということを示していると考えているが、物理学的な視点からすれば、絶対速度に対する言及であり、思想的な論理ではなく、実験事実に基づく論理である。



\begin{itembox}[l]{自然科学}
文章の主旨を要約せよ。\par

\ \ \ \ \ \ \ \ \ If you tried to pay for something with a piece of paper, you might run into some trouble. Unless, of course, the piece of paper was a hundred dollar bill. But what is it that makes that bill so much more interesting and valuable than other pieces of paper? After all, there's not much you can do with it. You can't eat it. You can't build things with it. And burning it is actually illegal. So what's the big deal?\par
\ \ \ \ \ \ \ \ \ Of course, you probably know the answer. A hundred dollar bill is printed by the government and designated as official currency, while other pieces of paper are not. But that's just what makes them legal. What makes a hundred dollar bill valuable, on the other hand, is how many or few of them are around. Throughout history, most currency, including the US dollar, was linked to valuable commodities and the amount of it in circulation depended on a government's gold or silver reserves. But after the US abolished this system in 1971, the dollar became what is known as fiat money, meaning not linked to any external resource but relying instead solely on government policy to decide how much currency to print. \par
\ \ \ \ \ \ \ \ \ Which branch of our government sets this policy? The Executive, the Legislative, or the Judicial? The surprising answer is: none of the above! In fact, monetary policy is set by an independent Federal Reserve System, or the Fed, made up of 12 regional banks in major cities around the country. Its board of governors, which is appointed by the president and confirmed by the Senate, reports to Congress, and all the Fed's profit goes into the US Treasury. But to keep the Fed from being influenced by the day-to-day vicissitudes of politics, it is not under the direct control of any branch of government. \par
\ \ \ \ \ \ \ \ \ Why doesn't the Fed just decide to print infinite hundred dollar bills to make everyone happy and rich? Well, because then the bills wouldn't be worth anything. Think about the purpose of currency, which is to be exchanged for goods and services. If the total amount of currency in circulation increases faster than the total value of goods and services in the economy, then each individual piece will be able to buy a smaller portion of those things than before. This is called inflation. On the other hand, if the money supply remains the same, while more goods and services are produced, each dollar's value would increase in a process known as deflation.\par
\ \ \ \ \ \ \ \ \ So which is worse? Too much inflation means that the money in your wallet today will be worth less tomorrow, making you want to spend it right away. While this would stimulate business, it would also encourage overconsumption, or hoarding commodities, like food and fuel, raising their prices and leading to consumer shortages and even more inflation. But deflation would make people want to hold onto their money, and a decrease in consumer spending would reduce business profits, leading to more unemployment and a further decrease in spending, causing the economy to keep shrinking. So most economists believe that while too much of either is dangerous, a small, consistent amount of inflation is necessary to encourage economic growth. \par
\ \ \ \ \ \ \ \ \ The Fed uses vast amounts of economic data to determine how much currency should be in circulation, including previous rates of inflation, international trends, and the unemployment rate. Like in the story of Goldilocks, they need to get the numbers just right in order to stimulate growth and keep people employed, without letting inflation reach disruptive levels. The Fed not only determines how much that paper in your wallet is worth but also your chances of getting or keeping the job where you earn it. 

\begin{thebibliography}{99}
\bibitem{sincho}
\begin{verbatim*}
https://www.ted.com/talks/doug_levinson_what_gives_a_dollar_bill_its_value/transcript
\end{verbatim*}
\end{thebibliography}

\end{itembox}


\begin{flushleft}
【解答】
\end{flushleft}

米ドルは固定相場制と政府の管轄から独立して,変動相場制およびFRBの統制下となった.FRBは膨大のデータによってデフレーションや過度なインフレーションを抑制して持続的な経済進歩を実行している.



\newpage


\begin{itembox}[l]{自然科学}

    次の文章は1912年の記事であり、現代的な世界観とは違うところがある。この文章から、エーテルとはどういった性質があるかについてまとめよ。\\
\ \ \ \ \ \ \ \ \ Throw a stone into a pool of water. A disturbance is immediately created, and little waves will radiate from the spot where the stone struck the water, gradually spreading out into enlarging circles until they reach the shores or die away. By throwing several stones in succession with varying intervals between them it would be possible to so arrange a set of signals that they would convey a meaning to one who is initiated, standing on the opposite side of the pool. The little waves are the vehicle which transmits the intelligence, and the water the medium in which the waves travel.\\
\ \ \ \ \ \ \ \ \ Wireless telegraph instruments are simply a means for creating and detecting waves in a great pool of ether.\\
\ \ \ \ \ \ \ \ \ Scientists suppose that all space and matter is pervaded with a hypothetical medium of extreme tenuity and elasticity, called luminiferous ether, or simply ether.\\
\ \ \ \ \ \ \ \ \ Although ether is invisible, odorless, and practically weightless, it is not merely the fantastic creation of speculative philosophers, but is as essential to our existence as the air we breathe and the food we eat. By imagining and accepting its reality, it is possible to explain and understand many scientific puzzles. The universe is a vast pool of ether. It is all-pervading. There is no void. It is diffused even among the molecules of which solid bodies are composed. The study of this substance is, perhaps, one of the most fascinating and important duties of the physicist.\\
\ \ \ \ \ \ \ \ \ Ninety million miles away from our earth is a huge flaming body of vapors and gases, called the sun. This seething mass of flame and heat furnishes us more than mere winter and summer and night and day, for we on this earth are not living on our own resources, and the real work of the world so necessary for even bare existence is accomplished by the energy of the sun stored up in coal, in plants and trees and mountain torrents.\\
\ \ \ \ \ \ \ \ \ Light is known to be vibrations of an extremely rapid period—electromagnetic waves, they are called. Heat can be shown to be of the same nature. Traveling at the rate of over 180,000 miles per second, these two great gifts of the sun come streaming continually down to us over the inconceivable distance of almost 100,000,000 miles. Both require a medium for their propagation. The ether supplies it. It is the substance with which the universe is filled. Incidentally it is also the seat of all electrical and magnetic forces.
\end{itembox}


\begin{flushleft}
【解答】
\end{flushleft}

庭師が花や実を育て上げるために、畑を耕して雑草を刈るように、人も精神の庭を耕し、誤謬や杞憂、雑念を刈り除いて初めて、正善で理性的かつ純粋な心情という花や実を育て上げるのだ。

\begin{flushleft}
【解説】
\end{flushleft}

Just as~ so...:~と同じように...\\
may S V(原型):SがVしますように(祈願文)\\
対比と比喩の多い文章です! 正確に文章を読み取りましょう! おそらくですが、文法事項にこだわった結果、文脈を読み違える事が起きているのかもしれません。\\
cultivating toward perfection \underline{the flowers and fruits} of right, useful, and pure thoughts.例えば下線部ですがこれは文章内には出ていない、文脈上の比喩で、「成果」に近い意味を持っています。


\newpage

\begin{itembox}[l]{パンデミック}
【問】から始まる段落を和訳せよ.\\
\ \ \ \ \ \ \ \ \ During the 1918 flu pandemic, which killed up to 50 million people worldwide, Americans got tired of being constrained and prematurely gave up on flu prevention measures. Two more waves of the flu pandemic hit the United States, resulting in more deaths.\\
\ \ \ \ \ \ \ \ \ While some parallels can be drawn between COVID-19 and the 1918 flu pandemic, looking to the past isn’t always a good barometer for when this pandemic might end because of the advanced knowledge and technology that exists today.\\
\ \ \ \ \ \ \ \ \ “We know exactly what we're supposed to do, and this is an advantage that people of the past did not necessarily have,” says Nükhet Varlik, associate professor of history at Rutgers University-Newark. “We have the vaccines. We have the public health regulations in place. We have the medical expertise, so we actually know what to do. So we're actually at an unprecedented advantage when we compare ourselves to past societies. We can actually do the right things. Whether we do the right things, that’s another question.”\\
\ \ \ \ \ \ \ \ \ Varlik says asking when the pandemic might end is misleading, fueling false hopes rather than focusing on trying to control and mitigate the pandemic.\\
\ \ \ \ \ \ \ \ \ “It will become endemic, but that doesn't mean that it cannot become pandemic again. So, it's kind of like a dance … it can be pandemic or epidemic or endemic, and it can change over time,” Varlik says. “I am pretty confident that COVID will continue to be epidemic in one part of the world for the foreseeable future … and, of course, with travel and other means, it can spill over to other places, to other countries. Until it's eliminated in the entire world, there is really no way of feeling safe from this disease.”\\
\ \ \ \ \ \ \ \ \ 【問】When epidemiologists will declare the pandemic over has a lot to do with how much disease a society is willing to accept and put up with, Navarro says. COVID-19 could become like the flu, killing tens of thousands of Americans every year, predominantly those in vulnerable medical categories.\\
\ \ \ \ \ \ \ \ \ “At some point, you just have to say to yourself, ‘You know, I live in the world. There are dangers in my world, infectious disease, car accidents.’ But you can't let that cripple you. Those things have always been there,” biology professor Sawyer says. “I certainly would never want to send a message that this is now yet another thing that people need to worry and have anxiety about once this becomes endemic. Instead, get your vaccine, get your flu vaccine, protect yourself and then go on with your life.”

\end{itembox}


\begin{flushleft}
【解答】
\end{flushleft}



\begin{flushleft}
【解説】
\end{flushleft}

\newpage

\begin{itembox}[l]{テンプレ}
【問】から始まる段落を和訳せよ.\\
\ \ \ \ \ \ \ \ \ UNICEF announced Sunday an emergency cash support effort for all public education teachers in Afghanistan for January and February, saying the move will allow continued access to education for all school-age girls and boys.\\
\ \ \ \ \ \ \ \ \ The European Union-funded payment, amounting to the equivalent of 100 dollars a month per teacher, will benefit an estimated 194,000 male and female teachers across the country who have not been paid for six months.\\
\ \ \ \ \ \ \ \ \ UNICEF said in a statement that the agency and partners have taken the initiative in recognition of the “crucial role” these teachers in Afghanistan are playing in the education of around 8.8 million enrolled in public schools.\\
\ \ \ \ \ \ \ \ \ “Following months of uncertainty and hardship for many teachers, we are pleased to extend emergency support to public school teachers in Afghanistan who have spared no effort to keep children learning,” said Mohamed Ayoya, UNICEF’s country representative.\\
\ \ \ \ \ \ \ \ \ Ayoya said UNICEF would need an additional 250 million dollars to be able to continue supporting public school teachers and called on donors to help the agency fund the initiative.\\
\ \ \ \ \ \ \ \ \ Since militarily taking over the country in mid-August, the Taliban have allowed women to resume work in health and education, and opened private and public universities to female education, while secondary school girls are also back in school in about a dozen of the 34 Afghan provinces.\\
\ \ \ \ \ \ \ \ \ The new Islamist rulers have pledged to allow all girls to return to school by late March, blaming delays on financial constraints and the time it takes to ensure that female students resume classes in accordance with Islamic Sharia law.\\
\ \ \ \ \ \ \ \ \ Relief agencies say humanitarian needs have skyrocketed in war-torn Afghanistan since the Taliban took power last year and U.S.-led international forces withdrew from the country.\\
\ \ \ \ \ \ \ \ \ The United States and other Western nations have halted nonhumanitarian funding to Afghanistan, amounting to 40\% of the country’s gross domestic product, and blocked the Taliban’s access to billions of dollars in Afghan central bank reserves, mostly held in the United States.\\
\ \ \ \ \ \ \ \ \ The restrictions have pushed the fragile Afghan economy to the brink of collapse, worsening the humanitarian crisis stemming from years of war and natural calamities.\\
\ \ \ \ \ \ \ \ \ The United Nations is warning that nearly 23 million people — about 55\% of the poverty-stricken country's population — face extreme hunger, with nearly 9 million a step away from famine.\\
\ \ \ \ \ \ \ \ \ Tomas West, the U.S. special envoy for Afghanistan, while speaking at the Munich Security Conference on Saturday, said Washington was also playing its role in ensuring Afghan girls return to schools next month.\\
\ \ \ \ \ \ \ \ \ “We have before the World Bank a proposal to extend roughly 180 million dollars in support of teacher stipends and in support of books and in support of refurbishment of buildings and so forth,” he said.\\
\ \ \ \ \ \ \ \ \ “But what do we need to see from the Taliban? We need to see them deliver on stated commitments to open and enroll women and girls in education countrywide… after Nowruz [the first day of Afghan new year] on March 20th,” West stressed.\\
\ \ \ \ \ \ \ \ \ 【問】No country has yet recognized the Taliban as the legitimate rulers of Afghanistan. Before considering the legitimacy issue, the global community wants the Islamist group to install an inclusive government in Kabul representing all Afghan ethnic groups, ensure women’s rights to education and work, and prevent terrorist groups from using Afghan soil for attacks against other countries

\end{itembox}


\begin{flushleft}
【解答】
\end{flushleft}



\begin{flushleft}
【解説】
\end{flushleft}

\newpage

\begin{itembox}[l]{テンプレ}



\end{itembox}


\begin{flushleft}
【解答】
\end{flushleft}



\begin{flushleft}
【解説】
\end{flushleft}



\end{document}